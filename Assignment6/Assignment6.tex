\documentclass[journal,12pt,twocolumn]{IEEEtran}

\usepackage{setspace}
\usepackage{gensymb}
\singlespacing
\usepackage[cmex10]{amsmath}

\usepackage{amsthm}

\usepackage{mathrsfs}
\usepackage{txfonts}
\usepackage{stfloats}
\usepackage{bm}
\usepackage{cite}
\usepackage{cases}
\usepackage{subfig}

\usepackage{longtable}
\usepackage{multirow}

\usepackage{enumitem}
\usepackage{mathtools}
\usepackage{steinmetz}
\usepackage{tikz}
\usepackage{circuitikz}
\usepackage{verbatim}
\usepackage{tfrupee}
\usepackage[breaklinks=true]{hyperref}
\usepackage{graphicx}
\usepackage{tkz-euclide}

\usetikzlibrary{calc,math}
\usepackage{listings}
    \usepackage{color}                                            %%
    \usepackage{array}                                            %%
    \usepackage{longtable}                                        %%
    \usepackage{calc}                                             %%
    \usepackage{multirow}                                         %%
    \usepackage{hhline}                                           %%
    \usepackage{ifthen}                                           %%
    \usepackage{lscape}     
\usepackage{multicol}
\usepackage{chngcntr}

\DeclareMathOperator*{\Res}{Res}

\renewcommand\thesection{\arabic{section}}
\renewcommand\thesubsection{\thesection.\arabic{subsection}}
\renewcommand\thesubsubsection{\thesubsection.\arabic{subsubsection}}

\renewcommand\thesectiondis{\arabic{section}}
\renewcommand\thesubsectiondis{\thesectiondis.\arabic{subsection}}
\renewcommand\thesubsubsectiondis{\thesubsectiondis.\arabic{subsubsection}}


\hyphenation{op-tical net-works semi-conduc-tor}
\def\inputGnumericTable{}                                 %%

\lstset{
%language=C,
frame=single, 
breaklines=true,
columns=fullflexible
}
\begin{document}


\newtheorem{theorem}{Theorem}[section]
\newtheorem{problem}{Problem}
\newtheorem{proposition}{Proposition}[section]
\newtheorem{lemma}{Lemma}[section]
\newtheorem{corollary}[theorem]{Corollary}
\newtheorem{example}{Example}[section]
\newtheorem{definition}[problem]{Definition}

\newcommand{\BEQA}{\begin{eqnarray}}
\newcommand{\EEQA}{\end{eqnarray}}
\newcommand{\define}{\stackrel{\triangle}{=}}
\bibliographystyle{IEEEtran}
\raggedbottom
\setlength{\parindent}{0pt}
\providecommand{\mbf}{\mathbf}
\providecommand{\pr}[1]{\ensuremath{\Pr\left(#1\right)}}
\providecommand{\qfunc}[1]{\ensuremath{Q\left(#1\right)}}
\providecommand{\sbrak}[1]{\ensuremath{{}\left[#1\right]}}
\providecommand{\lsbrak}[1]{\ensuremath{{}\left[#1\right.}}
\providecommand{\rsbrak}[1]{\ensuremath{{}\left.#1\right]}}
\providecommand{\brak}[1]{\ensuremath{\left(#1\right)}}
\providecommand{\lbrak}[1]{\ensuremath{\left(#1\right.}}
\providecommand{\rbrak}[1]{\ensuremath{\left.#1\right)}}
\providecommand{\cbrak}[1]{\ensuremath{\left\{#1\right\}}}
\providecommand{\lcbrak}[1]{\ensuremath{\left\{#1\right.}}
\providecommand{\rcbrak}[1]{\ensuremath{\left.#1\right\}}}
\theoremstyle{remark}
\newtheorem{rem}{Remark}
\newcommand{\sgn}{\mathop{\mathrm{sgn}}}
\providecommand{\abs}[1]{\left\vert#1\right\vert}
\providecommand{\res}[1]{\Res\displaylimits_{#1}} 
\providecommand{\norm}[1]{\left\lVert#1\right\rVert}
%\providecommand{\norm}[1]{\lVert#1\rVert}
\providecommand{\mtx}[1]{\mathbf{#1}}
\providecommand{\mean}[1]{E\left[ #1 \right]}
\providecommand{\fourier}{\overset{\mathcal{F}}{ \rightleftharpoons}}
%\providecommand{\hilbert}{\overset{\mathcal{H}}{ \rightleftharpoons}}
\providecommand{\system}{\overset{\mathcal{H}}{ \longleftrightarrow}}
	%\newcommand{\solution}[2]{\textbf{Solution:}{#1}}
\newcommand{\solution}{\noindent \textbf{Solution: }}
\newcommand{\cosec}{\,\text{cosec}\,}
\providecommand{\dec}[2]{\ensuremath{\overset{#1}{\underset{#2}{\gtrless}}}}
\newcommand{\myvec}[1]{\ensuremath{\begin{pmatrix}#1\end{pmatrix}}}
\newcommand{\mydet}[1]{\ensuremath{\begin{vmatrix}#1\end{vmatrix}}}
\numberwithin{equation}{subsection}
\makeatletter
\@addtoreset{figure}{problem}
\makeatother
\let\StandardTheFigure\thefigure
\let\vec\mathbf
\renewcommand{\thefigure}{\theproblem}
\def\putbox#1#2#3{\makebox[0in][l]{\makebox[#1][l]{}\raisebox{\baselineskip}[0in][0in]{\raisebox{#2}[0in][0in]{#3}}}}
     \def\rightbox#1{\makebox[0in][r]{#1}}
     \def\centbox#1{\makebox[0in]{#1}}
     \def\topbox#1{\raisebox{-\baselineskip}[0in][0in]{#1}}
     \def\midbox#1{\raisebox{-0.5\baselineskip}[0in][0in]{#1}}
\vspace{3cm}
\title{Assignment 6}
\author{Gautham Bellamkonda - CS20BTECH11017}
\maketitle
\newpage
\bigskip
\renewcommand{\thefigure}{\theenumi}
\renewcommand{\thetable}{\theenumi}
Download all
%\begin{lstlisting}
%https://github.com/GauthamBellamkonda/AI1103/tree/main/Assignment6/Codes
%\end{lstlisting} 
latex-tikz codes from 
\begin{lstlisting}
https://github.com/GauthamBellamkonda/AI1103/tree/main/Assignment6
\end{lstlisting}
\section{Problem}
Let $X_1, X_2, \ldots , X_n$ be a random sample of size $n \; ( \geq 2 ) $ from a distribution having the probability density function
\begin{align}
f(x;\theta) = 
\begin{cases}
\dfrac{1}{\theta} \exp\brak{-\dfrac{x}{\theta}} & x > 0, \\
0, & \text{otherwise,}
\end{cases}
\end{align}
where $\theta \in (0, \infty)$. Let $X_{(1)} = $ min $ \{ X_1, X_2, \ldots , X_n \} $ and $T = \sum_{i=1}^n X_i $. Then $E(X_{(1)} | T)$ equals $\ldots \ldots \ldots$
\begin{enumerate}[label = (\Alph*)]
\item $\dfrac{T}{n^2}$ \\
\item $\dfrac{T}{n}$ \\
\item $\dfrac{(n+1)T}{2n}$ \\
\item $\dfrac{(n+1)^2 T}{4n^2}$
\end{enumerate}

\section{Solution}
\textbf{Lehmann–Scheffé theorem :}

If $T$ is a complete sufficient statistic for $\theta$ and 
\begin{align}
\label{eqn 2.0.1}
E(g(T)) = \tau(\theta)
\end{align}
then $g(T)$ is the uniformly minimum-variance unbiased estimator (UMVUE) of $\tau(\theta)$.\\

We know that 
\begin{align}
T = \sum_{i=1}^{n} X_i
\end{align}
is a complete and sufficient statistic. By the law of total expectation, 
\begin{align}
\label{eqn 2.0.3}
E\brak{E(X_{(1)} | T )} = E(X_{(1)})
\end{align}
By Lehmann–Scheffé theorem, with
\begin{align}
\theta &= X_{(1)},\\ 
\tau(x) &= E(x),\\
g(T) &= E(X_{(1)} | T).
\end{align}
it follows from \eqref{eqn 2.0.3} that $E(X_{(1)} | T)$ is the UMVUE of $E(X_{(1)})$.
\begin{align}
\pr{X_{(1)} > x} &= \pr{X_1 > x}\ldots \pr{X_n > x}\\
&= (1-F_{X_{1}}(x))\ldots(1-F_{X_{n}}(x))\\
&= (1-F_{X_{1}}(x))^n \\
&= \exp\brak{-\frac{nx}{\theta}}\\
F_{X_{(1)}}(x) &= 1 - \exp\brak{-\frac{nx}{\theta}}\\
f_{X_{(1)}}(x) &= \frac{n}{\theta} \exp\brak{-\frac{nx}{\theta}}
\end{align}
Therefore, $X_{(1)}$ follows an exponential distribution with mean $\dfrac{\theta}{n}$.
\begin{align}
E(X_{(1)}) = \frac{\theta}{n}
\end{align}
Note that,
\begin{align}
E\brak{\frac{T}{n^2}} &= E\brak{\frac{\sum_{i=1}^n X_i}{n^2}}\\
&= \frac{E(\sum_{i=1}^n X_i)}{n^2}\\
&= \sum_{i=1}^n \frac{E(X_i)}{n^2}\\
&= \sum_{i=1}^n \frac{\theta}{n^2}\\
&= \frac{\theta}{n}\\
&= E(X_{(1)})
\end{align}
Therefore, by Lehmann–Scheffé theorem, with
\begin{align}
\theta &= X_{(1)},\\
\tau(x) &= E(x),\\
g(T) &= \frac{T}{n^2},
\end{align}
it follows that $\dfrac{T}{n^2}$ is UMVUE of $E(X_{(1)})$.\\

Since there exists a unique UMVUE for $E(X_{(1)})$, it follows that 
\begin{align}
E(X_{(1)} | T) = \frac{T}{n^2}
\end{align}
Hence, option A is correct.
\end{document}