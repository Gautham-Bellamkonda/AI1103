\subsection{Boolean Logic}
If A and B are two events such that $\pr{A} = \frac{1}{4}, \pr{B} = \frac{1}{2}$ and $\pr{A B} = \frac{1}{8}$. find $\pr{\text{not A and not B}}$.
\renewcommand{\theequation}{\theenumi}
\begin{enumerate}[label=\thesubsection.\arabic*.,ref=\thesubsection.\theenumi]
\numberwithin{equation}{enumi}

\item 
\begin{align}
A^{\prime}B^{\prime} &=  \brak{A+B}^{\prime}
\\
\implies \pr{A^{\prime}B^{\prime}} &=  \pr{\brak{A+B}^{\prime}} 
\\
&= 1 - \pr{A+B} 
\label{eq:axiom_sum_one}
\end{align}
\item 
\begin{align}
\because A+B &= A\brak{B+B^{\prime}} + B
\\
&= B \brak{A +1} + A B^{\prime}
\\
&=B + A B^{\prime}
\\
\implies \pr{A+B} &= \pr{B + A B^{\prime} }
\\
&=\pr{B}+\pr{ A B^{\prime} } 
\\
&\because B \brak{ A B^{\prime} } = 0
\label{eq:axiom_sum_two}
\end{align}
\item 
\begin{align}
A = A \brak{B+B^{\prime}} =  AB + AB^{\prime}
\label{eq:axiom_sum_A}
\end{align}
and 
\begin{align}
\brak{ AB}\brak{  AB^{\prime}} = 0, \because BB^{\prime} = 0
\label{eq:axiom_sum_AB0}
\end{align}
Hence, $AB$ and $AB^{\prime}$ are mutually exclusive and 
%
\begin{align}
\pr{A} = \pr{AB} + \pr{AB^{\prime}}
\\
\implies 
\pr{AB^{\prime}} =  \pr{A} - \pr{AB}
\label{eq:axiom_sum_ABp}
\end{align}
\item Substituting \eqref{eq:axiom_sum_ABp} in \eqref{eq:axiom_sum_two}, 
\begin{align}
\pr{A+B} &= \pr{A} + \pr{B} - \pr{AB} 
\label{eq:axiom_sum_AB}
\end{align}
\item Substituting \eqref{eq:axiom_sum_AB} in \eqref{eq:axiom_sum_one}
\begin{align}
\pr{A^{\prime}B^{\prime}} &=  1 - \cbrak{\pr{A} + \pr{B} - \pr{AB} }
\\
&= 1 - \brak{\frac{1}{4} + \frac{1}{2} - \frac{1}{8}}
\\
&= \frac{3}{8}
\label{eq:axiom_sum_final}
\end{align}
\end{enumerate}
\subsection{Independent Events}
\renewcommand{\theequation}{\theenumi}
\begin{enumerate}[label=\thesubsection.\arabic*.,ref=\thesubsection.\theenumi]
\numberwithin{equation}{enumi}



\item Prove that if $E$ and $F$ are independent events, then so are the events $E$ and $F^{\prime}$.\\
\solution  If $E$ and $F$ are independent,
\begin{align}
\pr{EF} = \pr{E}\pr{F}
\label{eq:axiom_indep}
\end{align}
From 
\eqref{eq:axiom_sum_AB0}
%
\begin{align}
\pr{EF^{\prime}} =  \pr{E} - \pr{EF}
\label{eq:axiom_indep_EFp}
\end{align}
Substituting from \eqref{eq:axiom_indep} in \eqref{eq:axiom_indep_EFp},
%
\begin{align}
\pr{EF^{\prime}} &=  \pr{E} \brak{1- \pr{F}}
&= \pr{E} \pr{F^{\prime}}
\label{eq:axiom_indep_EFp_ind}
\end{align}
%
\begin{align}
\because FF^{\prime} = 0, F + F^{\prime} = 1
\\
\implies \pr{F}+\pr{F^{\prime}} = 1
\label{eq:axiom_FFp}
\end{align}
By definition, from \eqref{eq:axiom_indep_EFp_ind}, we conclude that $E$ and $F^{\prime}$ are independent.
\item If A and B are two independent events, then the probability of occurrence of at least one of A and B is given by 1- $P(A^{\prime}) P(B^{\prime})$\\
\solution 
\begin{align}
\because (A+B)(A+B)^{\prime} = 0
\\
\implies 1 = \pr{A+B} + \pr{\brak{A+B}^{\prime}}
\\
\implies \pr{A+B} = 1 - \pr{A^{\prime}B^{\prime}} 
\\
= 1 - \pr{A^{\prime}}\pr{B^{\prime}} 
\end{align}
using the definition of independence.
\end{enumerate}
\subsection{Conditional Probability}
\renewcommand{\theequation}{\theenumi}
\begin{enumerate}[label=\thesubsection.\arabic*.,ref=\thesubsection.\theenumi]
\numberwithin{equation}{enumi}

\item Given that E and F are events such that P(E) = 0.6, P(F) = 0.3 and P(E  F) = 0.2, find $\pr{E|F}$ and $\pr{F|E}$\\
\solution By definition,
\begin{align}
\pr{E|F} = \frac{\pr{EF}}{\pr{F}} = \frac{0.2}{0.3} = \frac{2}{3}
\end{align}
%
Similarly,
\begin{align}
\pr{F|E} = \frac{\pr{EF}}{\pr{E}} = \frac{1}{3}
\end{align}


\item A fair die is rolled. Consider the events E =  (1, 3, 5), F = (2, 3) and G = (2, 3, 4, 5) Find\\
\begin{enumerate}
\item  $\pr{E|F}$ and $\pr{F|E}$
\item  $\pr{E|G}$ and  $\pr{F|E}$
\item   $\pr{\brak{E+F}|G}$ and  $\pr{EF|G}$ 
\end{enumerate}
\solution
%\input{./solutions/docq22.tex}

From the given information,
	\begin{align}
	\pr{E} &= \frac{3}{6} = \frac{1}{2} \\
	\pr{F} &= \frac{2}{6} = \frac{1}{3} \\	
	\pr{G} &= \frac{4}{6} = \frac{2}{3} \\	
	\pr{E F} &= \frac{1}{6}\\
	\pr{E G} &= \frac{2}{6}= \frac{1}{3}\\
	\pr{F G} &= \frac{2}{6}= \frac{1}{3}\\
	\pr{E F G} &= \frac{1}{6}
\end{align}
\begin{enumerate}
\item	
\begin{align}
	\pr{E|F} &= \frac{\pr{E F}}{\pr{F}}\\
	\pr{E|F} &= \frac{\frac{1}{6}}{\frac{1}{3}} = \frac{1}{2}\\
	\pr{F|E} &= \frac{\pr{F E}}{\pr{E}}\\
	\pr{F|E} &= \frac{\frac{1}{6}}{\frac{1}{2}} = \frac{1}{3}
	\end{align}

\item 	\begin{align}
	\pr{E|G} &= \frac{\pr{E G}}{\pr{G}}\\
	\pr{E|G} &= \frac{\frac{1}{3}}{\frac{2}{3}} = \frac{1}{2}\\
	\pr{G|E} &= \frac{\pr{G E}}{\pr{G}}\\
	\pr{G|E} &= \frac{\frac{1}{3}}{\frac{1}{2}} = \frac{2}{3}\\
	\end{align}
	
\item	%$\pr{\frac{E\cup F}{G}}$
\begin{multline}
\pr{E+F|G} = \frac{\pr{\cbrak{E+F}G}}{\pr{G}}
\\
 = \frac{\pr{EG + FG}}{\pr{G}}
\\
=\frac{\pr{EG} +\pr{ FG}- \pr{EFG}}{\pr{G}}
\\
 = \frac{3}{4}
\end{multline}
and 
\begin{align}
\pr{EF|G} = 
\frac{ \pr{EFG}}{\pr{G}}
 = \frac{1}{4}
\end{align}

\end{enumerate}


\end{enumerate}

